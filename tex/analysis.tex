\chapter{Аналитическая часть}

\section{Многопоточность}

Многопоточность - это форма распараллеливания или разделения работы для одновременной обработки \cite{intel-multithreading}. Вместо того, чтобы нагружать одно ядро большой нагрузкой, многопоточные программы разделяют работу на несколько программных потоков. Поток - некая сущность внутри процесса, получающая процессорное время для выполнения \cite{thread}. Процесс - это экземпляр выполняемой программы \cite{winForProf}.

Главное преимущество многопоточной обработки заключается в том, что она позволяет писать программы, которые работают очень эффективно благодаря возможности выгодно использовать время простоя, неизбежно возникающее в ходе выполнения большинства программ. 
Кроме того, параллельное выполнение нескольких потоков инструкций позволяет разделить выполнение трудоемких задач и уменьшить время их выполнения \cite{mult-in-apps}. 

Стек у каждого потока разный, но память для программного кода и куча разделяются среди всех потоков, функционирующих внутри одного процесса. 
Это позволяет потокам внутри одного процесса быстро взаимодействовать между собой, так как они разделяют одну и ту же память. Без должного контроля доступа к общей памяти может возникнуть состояние гонки, когда два или более потока пытаются одновременно изменить общие данные, что может привести к непредсказуемым результатам и ошибкам.

Для предотвращения состояний гонки часто применяются мьютексы и семафоры:
\begin{itemize}
	\item Определение Мьютекса.
	\item Опеределение семафора. 
\end{itemize}

\section{Нечеткая кластеризация с-средних}